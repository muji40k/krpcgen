\section*{Заключение}
\addcontentsline{toc}{section}{Заключение}

При выполнении курсовой работы был проведен анализ протоколов ONC PRC и XDR, а
также их реализация в ядре операционной системы Linux, были выделены и
рассмотрены структуры для описания предоставляемых (запрашиваемых) процедур и
соответствующие фунции ядра для запуска сервера и выполнения запроса. Были
разработаны алгоритмы для преобразования спецификации на языке RPCL в поток
токенов, для его последующего преобразования в поток определений и генерации на
его основе кода загружаемых модулей ядра. Была разработан минимальный макет
взаимодействия загружаемых модулей ядра, и утилита, способная генерировать код
загружаемых модулей ядра, взаимодействующих по модели клиент сервер согласно
протоколу ONC RPC на основе файла спецификации на языке RPCL, подобно
приложению \code{rpcgen} из пространства пользователя.

Проведенное исследование показало, что разработанное приложение полностью
соответствует техническому заданию, а именно позволяет генерировать код
загружаемых модулей ядра, взаимодействующих по сетевой модели, согласно
протоколу ONC RPC и файлу спецификации на языке RPCL, а также предоставляет
возможность конфигурации полученных модулей, а именно указание порта и ip
адреса сервера, количества потоков обработчиков и используемой версии.

