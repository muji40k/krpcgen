\leftsection{Конструкторская часть}

\vspace{-1\baselineskip}

\subsection{Требования к программе}

К разрабатываемому приложению выдвигаются следующие требования:
\begin{itemize}
    \item поддержка и разбор спецификации на языке rpcl;
    \item возможность раздельной загрузки и использования модулей разных
          программ;
    \item возможность конфигурирования порта и адреса модуля сервера;
    \item возможность использования предоставляемых процедур из сторонних
          модулей.
\end{itemize}

\subsection{Схема алгоритма программы}

Алгоритм работы приложения может быть разбит на следующие этапы, представленные
на рисунке~\ref{fig:main_alg}.

\begin{figure}[!h]
    \centering
    \def\svgwidth{0.18\textwidth}
    \input{main_alg.pdf_tex}
    \caption{Декомпозиция алгоритма работы программы}
    \label{fig:main_alg}
\end{figure}

\subsubsection{Преобразование текста спецификации}

Для обеспечения единообразия разбора определений описанного сервиса код
спецификации необходимо преобразовать в набор заранее описанных лексических
единиц --- токенов. Согласно структуре языка, описанной в
листингах~\ref{lst:xdrl} и~\ref{lst:rpcl}, можно выделить следующие группы
токенов:

\begin{itemize}
    \item идентификатор;
    \item ключевое слово (\code{"const"}, \code{"case"}, \code{"switch"},
          \code{"default"}, \code{"typedef"}, \code{"program"},
          \code{"version"}, \code{"procedure"});
    \item скобки (круглые \code{"()"}, фигурные \verb+"{}"+,
          квадратные \code{"[]"}, треугольные \code{"<>"});
    \item идентификатор типа (\code{"void"}, \code{"unsigned"},
          \code{"integer"}, \code{"hyper"}, \code{"float"}, \code{"double"},
          \code{"boolean"}, \code{"quadruple"}, \code{"string"},
          \code{"opaque"}, \code{"pointer"}, \code{"enum"}, \code{"struct"},
          \code{"union"});
    \item разделитель (точка с запятой \code{";"}, двоеточие \code{":"},
          запятая \code{","});
    \item оператор присвоения \code{"="};
    \item целочисленный литерал;
    \item комментарий.
\end{itemize}

На рисунках~\refrange{fig:lexer_alg1}{fig:lexer_alg2} представлен алгоритм
преобразования текста спецификации в набор токенов. Подразумевается, что
алгоритму доступен массив набора правил для сопоставления текущего символа.
Принцип работы такого набора правил сводится к последовательному сравнению
поставляемых символов с заранее заданным массивом или проверке его
принадлежности определенному подмножеству символов алфавита. В качестве
пропускаемых символов выступают все пробельные символы.

Был определен следующий приоритет групп токенов: разделители, скобки, литералы,
ключевые слова, идентификаторы типа, операторы, идентификаторы, комментарии.

\clearpage

\vspace*{\fill}
\begin{figure}[!h]
    \centering
    \def\svgwidth{\textwidth}
    \input{lexer_alg1.pdf_tex}
    \caption{Алгоритм лексического анализа текста спецификации (часть 1)}
    \label{fig:lexer_alg1}
\end{figure}
\vspace{\fill}
\clearpage

\vspace*{\fill}
\begin{figure}[!h]
    \centering
    \def\svgwidth{0.7\textwidth}
    \input{lexer_alg2.pdf_tex}
    \caption{Алгоритм лексического анализа текста спецификации (часть 2)}
    \label{fig:lexer_alg2}
\end{figure}
\vspace{\fill}
\clearpage

\subsubsection{Разбор определений}

В соответствии с листингами~\ref{lst:rpcl} и~\ref{lst:xdrl}, спецификация может
состоять из 3 видов определений:
\begin{enumerate}
    \item определение константы;
    \item определение типа;
    \item определение программы.
\end{enumerate}

Алгоритм преобразования потока токенов в поток определений представлен на
рисунках~\refrange{fig:parser_alg}{fig:parser_procedure_args_alg}

\begin{figure}[!h]
    \centering
    \def\svgwidth{\textwidth}
    \input{parser_alg.pdf_tex}
    \caption{Алгоритм разбора токенов}
    \label{fig:parser_alg}
\end{figure}

\clearpage

\vspace*{\fill}
\begin{figure}[!h]
    \centering
    \begin{minipage}{0.63\textwidth}\centering
        \def\svgwidth{\linewidth}
        \input{parser_const_alg.pdf_tex}
    \end{minipage}
    \hfill
    \begin{minipage}{0.35\textwidth}\centering
        \def\svgwidth{\linewidth}
        \input{parser_typedef_alg.pdf_tex}
    \end{minipage}
    \caption{Алгоритм разбора константы и \code{typedef}}
    \label{fig:parser_const_typedef_alg}
\end{figure}
\vfill
\begin{figure}[!h]
    \centering
    \begin{minipage}{0.6\textwidth}\centering
        \def\svgwidth{\linewidth}
        \input{parser_enum_alg.pdf_tex}
    \end{minipage}
    \hfill
    \begin{minipage}{0.35\textwidth}\centering
        \def\svgwidth{\linewidth}
        \input{parser_enum_body_alg.pdf_tex}
    \end{minipage}
    \caption{Алгоритм разбора \code{enum}}
    \label{fig:parser_enum_alg}
\end{figure}
\vfill

\clearpage

\vspace*{\fill}
\begin{figure}[!h]
    \centering
    \begin{minipage}{0.6\textwidth}\centering
        \def\svgwidth{\linewidth}
        \input{parser_struct_alg.pdf_tex}
    \end{minipage}
    \hfill
    \begin{minipage}{0.35\textwidth}\centering
        \def\svgwidth{\linewidth}
        \input{parser_struct_body_alg.pdf_tex}
    \end{minipage}
    \caption{Алгоритм разбора \code{struct}}
    \label{fig:parser_struct_alg}
\end{figure}
\vfill
\begin{figure}[!h]
    \centering
    \begin{minipage}{0.55\textwidth}\centering
        \def\svgwidth{\linewidth}
        \input{parser_union_alg1.pdf_tex}
    \end{minipage}
    \hfill
    \begin{minipage}{0.4\textwidth}\centering
        \def\svgwidth{\linewidth}
        \input{parser_union_alg2.pdf_tex}
    \end{minipage}
    \caption{Алгоритм разбора \code{union} (часть 1)}
    \label{fig:parser_union_alg1}
\end{figure}
\vfill

\clearpage

\vspace*{\fill}
\begin{figure}[!h]
    \centering
    \def\svgwidth{0.6\textwidth}
    \input{parser_union_alg3.pdf_tex}
    \caption{Алгоритм разбора \code{union} (часть 2)}
    \label{fig:parser_union_alg2}
\end{figure}
\vfill
\begin{figure}[!h]
    \centering
    \def\svgwidth{\textwidth}
    \input{parser_union_body_alg.pdf_tex}
    \caption{Алгоритм разбора тела \code{union}}
    \label{fig:parser_union_body_alg}
\end{figure}
\vfill

\clearpage

\vspace*{\fill}
\begin{figure}[!h]
    \centering
    \def\svgwidth{\textwidth}
    \input{parser_program_alg.pdf_tex}
    \caption{Алгоритм разбора программы}
    \label{fig:parser_program_alg}
\end{figure}
\vfill
\begin{figure}[!h]
    \centering
    \def\svgwidth{0.4\textwidth}
    \input{parser_program_body_alg.pdf_tex}
    \caption{Алгоритм разбора тела программы}
    \label{fig:parser_program_body_alg}
\end{figure}
\vfill

\clearpage

\vspace*{\fill}
\begin{figure}[!h]
    \centering
    \def\svgwidth{\textwidth}
    \input{parser_version_alg.pdf_tex}
    \caption{Алгоритм разбора версии}
    \label{fig:parser_version_alg}
\end{figure}
\vfill
\begin{figure}[!h]
    \centering
    \def\svgwidth{0.4\textwidth}
    \input{parser_version_body_alg.pdf_tex}
    \caption{Алгоритм разбора тела версии}
    \label{fig:parser_version_body_alg}
\end{figure}
\vfill

\clearpage

\vspace*{\fill}
\begin{figure}[!h]
    \centering
    \def\svgwidth{\textwidth}
    \input{parser_procedure_alg.pdf_tex}
    \caption{Алгоритм разбора процедуры}
    \label{fig:parser_procedure_alg}
\end{figure}
\vfill
\begin{figure}[!h]
    \centering
    \def\svgwidth{0.5\textwidth}
    \input{parser_procedure_args_alg.pdf_tex}
    \caption{Алгоритм разбора тела процедуры}
    \label{fig:parser_procedure_args_alg}
\end{figure}
\vfill

\clearpage

\subsubsection{Генерация кода модулей}

Алгоритм генерации модулей зависит от макета, который будет разработан в
разделе~\ref{sec:prototype}, и потому не будет приведен в качестве схемы. Общая
идея заключается в создании отдельного модуля для каждой программы. Между собой
модули будут разделять определения констант и типов.


