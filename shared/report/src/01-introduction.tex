\section*{Введение}
\addcontentsline{toc}{section}{Введение}

ONC RPC (Open Network Computing Remote Procedure Call) --- реализация системы
удаленного вызова процедур, разработанная как часть сетевой файловой системы
NFS~\cite{rfc1094}. Из-за широкого распространения \cite{rfc5531}, для
технология была разработана утилита генерации кода клиента и сервера на основе
языка RPCL (Remote Procedure Call Language)~\cite{rfc5531} --- \code{rpcgen}.
Однако, данная программа предназначена исключительно для пространства
пользователя. Целью данной работы является разработка приложения для генерации
кода пары загружаемых модулей ядра Linux, реализующих модель клиент-сервер и
позволяющие использовать технологию ONC RPC согласно спецификации на языке RPCL
в ядре операционной системы, подобно \code{rpcgen}.

Задачи:
\begin{enumerate}
    \item провести анализ протокола RPC;
    \item рассмотреть структуры и функции ядра, реализующие выбранную
          технологию;
    \item определить требуемый функционал приложения и разработать алгоритм
          работы как приложения-генератора, так и соответствующих загружаемых
          модулей;
    \item проанализировать результаты работы разработанного приложения.
\end{enumerate}

\pagebreak

