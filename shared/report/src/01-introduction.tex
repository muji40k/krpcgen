\section*{Введение}
\addcontentsline{toc}{section}{Введение}

ONC RPC --- реализация системы удаленного вызова процедур, разработанная как
часть сетевой файловой системы NFS~\cite{rfc1094}. Являясь широко используемым
инструментом~\cite{rfc5531}, технология обладает утилитой для генерации кода
клиента и сервера на основе языка RPCL~\cite{rfc5531} --- \code{rpcgen}.
Однако, данная программа предназначена исключительно для пространства
пользователя.

Целью данной работы является разработка приложения для генерации кода пары
загружаемых модулей ядра Linux, реализующих клиент и сервер, взаимодействующие
по протоколу ONC RPC согласно спецификации на языке RPCL.

Задачи:
\begin{enumerate}
    \item провести анализ протокола RPC;
    \item рассмотреть структуры и функции ядра, реализующие выбранную
          технологию;
    \item определить требуемый функционал приложения и разработать алгоритм
          работы как приложения-генератора, так и соответствующих загружаемых
          модулей;
    \item проанализировать результаты работы разработанного приложения.
\end{enumerate}

\pagebreak

